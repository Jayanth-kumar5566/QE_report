\section{List of pulications during PhD}
\begin{table}[H]
	\centering
	{\scriptsize 
\begin{tabular}{|l|l|l|}
	\hline
	No. &
	Publication &
	Type \\ \hline
	1 &
	\begin{tabular}[c]{@{}l@{}}Aogáin, M. M.; Lau, K. J. X.; Cai, Z.; \textbf{Narayana, J. K.}; Purbojati, R. W.; Drautz-Moses, D. I.; Gaultier, N. E.; \\ Jaggi, T. K.; Tiew, P. Y.; Ong, T. H.; Koh, M. S.; Hou, A. L. Y.; Abisheganaden, J. A.; Tsaneva-Atanasova, K.; \\ Schuster, S. C. \& Chotirmall, S. H.\\ Metagenomics Reveals a Core Macrolide Resistome Related to Microbiota in Chronic Respiratory Disease \\ American Journal of Respiratory and Critical Care Medicine\\ \\ Role: In this manuscript, I performed a co-occurrence analysis to uncover gene-microbial associations.\\  This network inference was performed using custom scripts in R by implementing Generalised Boosted \\ Linear Models (GBLM), with ReBoot (a randomization technique). The networks were visualized using \\ Cytoscape which was accessed through the CyRest API in python3.\end{tabular} &
	Original Article \\ \hline
	2 &
	\begin{tabular}[c]{@{}l@{}}Poh, T. Y.; Tiew, P. Y.; Lim, A. Y. H.; Thng, K. X.; Ali, N. A. B. M.; \textbf{Narayana, J. K.}; Aogáin, M. M.; Tien, Z.;\\ Chew, W. M.; Chan, A. K. W.; Keir, H. R.; Dicker, A. J.; Hassan, T. M.; Xu, H.; Tee, A. K.; Ong, T. H.; \\ Koh, M. S.; Abisheganaden, J. A.; Chalmers, J. D. \& Chotirmall, S. H.\\ Increased Chitotriosidase Is Associated With Aspergillus and Frequent Exacerbations in South-East \\ Asian Patients With Bronchiectasis, Chest\\ \\ Role: In this manuscript, I performed all the analysis which involved implementation of logistic \\ regression models, random forest, power calculation, optimum cut-off calculation and other statistical\\ analysis. A Nested cross-validation approach using Leave one out cross-validation was implemented\\ in combination with SMOTE (Synthetic Minority Oversampling Technique), a technique that addresses\\ class imbalance, a typical problem experienced in analysing biological datasets to train the random \\ forest model and its model evaluation.\end{tabular} &
	Original Article \\ \hline
	3 &
	\begin{tabular}[c]{@{}l@{}}Tiew, P. Y.; Ko, F. W. S.; \textbf{Narayana, J. K.}; Poh, M. E.; Xu, H.; Neo, H. Y.; Loh, L.-C.; Ong, C. K.;\\ Aogáin, M. M.; Tan, J. H. Y.; Kamaruddin, N. H.; Sim, G. J. H.; Lapperre, T. S.; Koh, M. S.;\\ Hui, D. S. C.; Abisheganaden, J. A.; Tee, A.; Tsaneva-Atanasova, K. \& Chotirmall, S. H.\\ “High-Risk” Clinical and Inflammatory Clusters in COPD of Chinese Descent, Chest\\ \\ Role: In this manuscript, I performed an analysis involving Feature selection, Clustering,\\ Cluster characterization and followed by cluster validation. Also, I implemented a decision tree \\ model to predict the cluster membership from baseline data. The clustering was performed after\\ data transformation and embedding patients in a euclidean space using hierarchical clustering. \\ These clusters were then tested for robustness and stability. RDA (Regularized discriminant analysis)\\ was implemented to validate the clusters. The model parameters were tuned to an optimal value \\ using Kernal density estimation (KDE). Further, the model accuracy was calculated using \\ Leave One Out Cross Validation approach. CART based decision trees were implemented to predict \\ these clusters.\end{tabular} &
	Original Article \\ \hline
	4 & 
	\begin{tabular}[c]{@{}l@{}}
	Nur A'tikah Binte Mohamed Ali, Fransiskus Xaverius Ivan, Micheál Mac Aogáin,\\ \textbf{Jayanth Kumar Narayana}, Shuen Yee Lee, Chin Leong\\ Lim \& Sanjay H. Chotirmall\\ The Healthy Airway Mycobiome in Individuals of Asian Descent, Chest\\ \\ Role: In this letter, I performed standard statistical analysis to compare\\ between the healthy and diseased mycobiome.\end{tabular} &
	Research Letter \\ \hline
\end{tabular}
}
\end{table}