\chapter*{Future works}
\addcontentsline{toc}{chapter}{Future works}

The developed and validated interactome framework used in first chapter to show that interactions are important than individual microbes and used in second chapter to show dysregulated lung-gut axis in high-risk bronchiectasis is based on “Graph theory”, a mathematical theory that study graphs as basic structures to model pairwise relation of nodes as points. Besides, mathematics also provides a generalisation of Graphs through “Simplicial complexes” from the field of Algebraic topology. A simplicial complex is a mathematical structure that models pairwise relation of simplices (generalisation of nodes), which captures complex relationships as points, lines, triangles and their n-dimensional counterparts. Given the advantages of these graph theory methods in capturing the interactome and thus providing deeper insights into disease prognosis and pathogenesis; as my next step, I would like to further improve the interactome framework using simplicial-complexes and methods from the field of topology. I am concurrently working with Asst. Prof. Xia Kelin to achieve this. In chapter1, we have shown that the post-exacerbation (post-antibiotic) interactome is predictive of time to next exacerbation with a GRsq (Generalised R Squared) of 55\%,
and I believe that with the introduction of topological based concepts which can capture the inherent structure of the network, could lead to further increase in this accuracy. Additionally, I also plan on implementing powerful prediction models that are based on machine learning to further increase the accuracy to predict clinical outcomes, advancing the field of precision medicine. Further, I am planning to submit an ERJMethods paper detailing methods such as Similarity Network Fusion (SNF) to integrate datasets.