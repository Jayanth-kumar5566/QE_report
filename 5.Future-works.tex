\chapter*{Future works}
\addcontentsline{toc}{chapter}{Future works}

The developed and validated interactome framework is based on “Graph theory”, a mathematical theory that study graphs as basic structures to model pairwise relation of nodes as points. Besides, mathematics also provides a generalisation of Graphs through “Simplicial complexes” from the field of Algebraic topology. A simplicial complex is a mathematical structure that models pairwise relation of simplices (generalisation of nodes), which captures complex relationships as points, lines, triangles and their n-dimensional counterparts. Given that interactions of nodes (microbes) are beneficial than isolated microbes in studying clinical outcomes such as exacerbations and antibiotic action; as my next step, I would like to test the hypothesise that the simplicial complex framework is more beneficial than the interactome framework, to study important clinical outcomes. We have shown that the post-exacerbation (post-antibiotic) interactome is predictive of time to next exacerbation with a GRsq (Generalised R Squared) of 55\%, if the proposed hypothesis were true, this could lead to increased accuracy of prediction. Secondly, I also plan on implementing powerful prediction models that are based on machine learning to further increase the accuracy to predict clinical outcomes, advancing the field of precision medicine. Further, I am planning to submit an ERJMethods paper detailing methods such as Similarity Network Fusion (SNF) to integrate datasets.