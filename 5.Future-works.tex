\chapter*{Future works}
\addcontentsline{toc}{chapter}{Future works}

The developed and validated interactome framework used in first chapter to show that interactions are important than individual microbes and used in second chapter to show dysregulated lung-gut axis in high-risk bronchiectasis is based on ``Graph theory", a mathematical theory that study graphs as basic structures to model pairwise relation of nodes as points. Besides, mathematics also provides a generalisation of Graphs through ``Simplicial complexes'' from the field of Algebraic topology. A simplicial complex is a mathematical structure that models pairwise relation of simplices (generalisation of nodes), which captures complex relationships as points, lines, triangles and their n-dimensional counterparts. Given the advantages of these graph theory methods in capturing the interactome and thus providing deeper insights into disease prognosis and pathogenesis; as my next step, I would like to further improve the interactome framework using simplicial-complexes and methods from the field of topology. I am concurrently working with Asst. Prof. Xia Kelin to achieve this. In chapter1, I have shown that the post-exacerbation (post-antibiotic) interactome is predictive of time to next exacerbation with a GRsq (Generalised R Squared) of 55\%,
and I believe that with the introduction of topological based concepts which can capture the inherent structure of the network, could lead to further increase in this accuracy. Additionally, I also plan on implementing powerful prediction models that are based on machine learning to further increase the accuracy to predict clinical outcomes, advancing the field of precision medicine. Further, I am planning to submit an ERJMethods paper detailing methods such as Similarity Network Fusion (SNF) to integrate datasets.

However, I also acknowledge that the analytical and mathematical methods developed during my PhD to predict microbial interactions (`microbial association networks') are predictive in nature and needs validation. I have tried validating these interactions by: 1) experimentally validating \emph{pseudomonas-aspergillus} interaction, one of the many interactions and 2) Showing that interactomes derived using metagenomics and targetted sequencing have a significant match of 89.9\%. Nevertheless, this doesn't completely validate the network inference framework and does not provide insights on how the mathematical values derived using these methods precisely translate into biological phenomenon. Therefore, I find this to be integral and I plan on validating and improving the network-inference framework using mathematical modelling and experimental work. The mathematical modelling will involve using real experimental data which I generate, to understand and explain the experimentally observed phenomenon. This will thus provide explain-ability and help enhance the microbial network inference algorithms. This work will be done at University of Exeter.

In summary relating back to my objectives, 
\begin{enumerate}
	\item I have successfully accomplished the first objective, by developing integrative strategies for microbiome datasets(Integrative Microbiomics) and to predict relationships between microbes (Co-occurrence analysis). This will serve as Chapter1 of my thesis. 
	\item I am presently working on predict exacerbation class using microbiomes derived from metagenomic datasets in the CAMEB2 (Cohort of Asian Matched European Bronchiectasis 2) cohort. This prediction model will use machine learning techniques and improve further on the idea of using interactions to predict clinical outcomes. This work will also include and enhance results from my lab-rotation which focused on ricci curvature and ricci flow on complex networks. This will serve as Chapter2 of my thesis.
	\item Thirdly, I will accomplish my final objective at University of Exeter, where I will mathematically model microbial communities taking into account microbial pathways and signalling networks. This will power our microbial airway interaction models and provide therapeutic and pharmaceutical relevance to their in vivo relationships. This work will serve as Chapter3 of my thesis.
\end{enumerate}