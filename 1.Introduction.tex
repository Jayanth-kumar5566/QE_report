\chapter*{Introduction}
\addcontentsline{toc}{chapter}{Introduction}

Understanding how individual people respond to medical therapy is a crucial facet of improving the odds ratio that interventions will have a positive impact. Reducing the non-responder rate for intervention or reducing complications associated with a particular treatment is the next stage of for any medical advance. The Precision Medicine Initiative, launched in January 2015, set the stage for enhanced collaboration between researchers and medical professionals to develop next-generation techniques to aid patient treatment and recovery and increased opportunity for impactful pre-emptive care. The microbiome plays a crucial role in health and disease, as it has implications in endocrinology, physiology, and even neurology, altering the outcome of many disease states, including its ability to augments drug response and tolerance.

Therefore, in precision medicine, the focus is on the identification of effective approaches for particular patients based on their genetic, lifestyle and environmental factors. Asian and European phenotypes of respiratory disease and infection are unique and therefore require such precision. While such approaches have been successfully employed to investigate contrasting clinical phenotypes; and by disease trajectories, little is known about `precision through microbes'. Precision medicine can be applied to the lung microbiome that includes both bacteria and fungi and their associated metabolic states. These `microbial fingerprints' permit patient stratification and we can identify particular disease phenotypes associated with clinical outcomes potentially amenable to precision and individualised intervention. It is clear that our microbes tell us something about disease, something representing a potential target for clinical intervention. During my PhD, I aim to extend and explore this in the context of pulmonary microbiome and lung-diseases. 

I aim to accomplish this by fulfilling the below objectives: 
\begin{enumerate}
	\item Model mathematically microbiome and mycobiome populations and their interactions across a range of pulmonary disease states: I plan to achieve this by developing computational approaches to identify mathematically significant co-operative and competitive relationships within and between species. There is also a scope for spatio-temporal modelling of the microbiome and mycobiome across various sites in the body, and study its effect on lung diseases.
	\item Develop mathematical models and tools using machine learning techniques to clinical settings in diagnosis, prognosis and predicting disease progression of patients using the pulmonary microbiome in respiratory disease states.
	\item Finally, using microbial metabolic datasets, I will further extend the developed models to take into account the affected pathways and signalling networks. This will also power our microbial airway interaction models and provide therapeutic and pharmaceutical relevance to their in vivo relationships. 
\end{enumerate}

At this point, during the submission of my Qualifying Examination report, I have accomplished the first part of my objective addressing and developing tools that capture interactions between microbes. This work is further reported in this document as two chapters with the first chapter focusing on the developed method ``Integrative microbiomics'': codified as an online webtool (\url{https://integrative-microbiomics.ntu.edu.sg/})  which when applied to pulmonary microbiome in bronchiectasis reveals a disrupted interactome. This chapter is written as a manuscript and has been submitted to Nature Medicine for publication as an original article. The second chapter of this report focuses on the application of the developed method to evaluate the `lung-gut' axis in bronchiectasis. We report the first study of the `lung-gut' axis in bronchiectasis and show a microbial dysregulation of the ‘lung-gut’ axis in high-risk bronchiectasis. 