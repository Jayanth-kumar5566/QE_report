\section{Discussion}

In this chapter, we describe to the best of our knowledge the first description of the `lung-gut' axis in bronchiectasis.
We demonstrate that microbial communities significantly overlap for fungi but not bacteria between the lung and gut compartments. We also show that fungal and bacterial diversity were respectively highest in the lung and gut. In the previous chapter of my PhD thesis, I had shown the advantages of integrating multiple microbiome datasets from the same site (i.e. lung). In this chapter, I extend this further to include multiple microbiome datasets from multiple sites (i.e. lung and gut). We demonstrate that integration of microbiome and mycobiome from lung and gut identifies `high-risk' patient group characterised by severe clinical and radiological bronchiectasis, including exacerbations. Upon assessment, these patients exhibit significant increases in gut \textit{Candida}, lung \textit{Fusobacteria} along with `lung-gut' interaction  and correlation of \textit{Streptococcus} between lung and gut, suggestive of a dysregulated `lung-gut' axis. Taken together, these findings reveal that `lung-gut' axis is crucial in bronchiectasis, providing fresh avenues to better understand bronchiectasis pathogenesis and its progression. 

Recently, with increasing studies supporting the concept of `lung-gut' axis, many researchers believe in the existence of `lung-gut' axis and its effectual role in disease progression and treatment. Microbes from the lung and gut can interact in complex ways, through immune modulation, production of microbial ligands, microbial metabolites and migratory immune cells or even microbes, through processes such as micro aspiration. In this study, we show that \textit{Streptococcus} in lung is correlated with \textit{Streptococcus} in gut in `high-risk' bronchiectasis patients, which potentially can be explained by a dysregulated lung-gut axis which leads to migration of bacteria between the lung and gut. Moreover, we also show an increased lung-gut microbial interaction in `high-risk' bronchiectasis patients further supplementing the theory of dysregulated axis. Results from this study shows that `lung-gut' is important in bronchiectasis and if established, can provide effective avenues for treatment. Hence, further studies with improved interventional experiments are needed to elucidate the role of microbiota in the lung-gut cross talk in bronchiectasis.  

Limitations of this study include the cross-sectional study design of this cohort, which we use to predict the dynamic processes of the lung-gut axis. However, we partially try to overcome this by assuming that the lung-gut axis is affected during disease states. We then use the groups defined by the integrated microbiome that separate the high and low risk disease states to show a dysregulated lung-gut axis. Additionally, we use a small sample size of 57 patients from a single site to infer a general phenomenon. Although determining microbiome and mycobiome from two sites in each patients partially overcomes the problem of small sizes by increasing the power \cite{Li2018}, this could still be affected by geographical conditions. Furthermore, fungal ITS sequencing approaches are challenged by under-developed reference databases and could lead to biased OTU picking. Finally, while observed data suggests a potential dysregulation of the lung-gut axis, other factors may drive these observed phenomenon. Hence, further interventional longitudinal studies are needed to experimentally validate the existence of lung-gut interaction and its effectual role in diseases.
