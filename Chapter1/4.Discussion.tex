\section{Discussion}
In my master’s thesis, we presented to the best of our knowledge, the first ‘multi-biome’ analysis using ‘integrative microbiomics’ combining bacterial, viral, and fungal communities in individual patients. Using a modified weighted-SNF, we identified frequent exacerbators with high precision and classified microbes within an ‘interactome’ as ‘busy’, ‘influential’ and/or ‘critical’. Frequent exacerbators exhibited antagonistic interactomes. In my present PhD thesis, we extended this further by preforming a longitudinal assessment over an exacerbation. This reveal disrupted interactomes, undetectable by assessing microbial identity alone. By use of simulation followed by confirmatory validation, we demonstrate interactome’s clinical relevance for modelling microbiome re-configuration in response to antibiotic exposure. Validation of interactomes was achieved by metagenomics which identifies a cluster that exhibits, a similar high-risk of exacerbation phenotype as identified from the derivation cohort. Further, interactome analysis of the high-risk cluster derived using the metagenomic validation cohort validates 89.9\% of the interactions. We also, provide microbiological evidence in support of our interactome approach by demonstrating variable interaction between \emph{P. aeruginosa} and \emph{A. fumigatus} using cluster-specific clinical isolates. We then assessed the clinical applicability of the interactome by modelling time to next exacerbation using interactions and individual taxa as features. Interestingly, we find a major increase in the accuracy of prediction when using interactions, in contrast to individual taxa. Taken together, our findings reveal a novel aspect of the functional microbiome with potential implications for the use of antibiotics in clinical practice.

It is well recognised in bronchiectasis, that patients improve despite receiving antibiotics not necessarily targeting their dominant pathogen. However, the conventional model where targeting bacteria with antibiotics reduces bacterial load, accompanying inflammation and therefore, exacerbation risk, which, in turn, alleviates symptoms and improves clinical outcomes; fails to explain this. If the interactome framework were true, then this could offer explanations of unexplained clinical observations of antibiotic use and help treat exacerbations. Results from this study show that interactions are more predictive than individual taxa of time to next exacerbation and better explain exacerbation, in support of the hypothesis. The airway microbiome (and its accompanying interactome) is likely a critical predictor of antibiotic treatment response and provides a theoretical basis for understanding several phenomena associated with antibiotics that remain unexplained clinically including antimicrobial responses in apparently resistant organisms. Manipulating microbiomes by means other than antibiotics are being explored and the effect of probiotics on the interactome should be considered.

The value of data integration using SNF for multidimensional datasets (such as multi-omics) in airways disease such as COPD has been demonstrated, however, these methods have not been previously applied to microbiome integration \cite{Li2018}. Conventional SNF is not optimized for biological systems such as multi-kingdom microbiomes where dynamism and potential dominance of one kingdom over the others needs to be considered. Employing a weighted SNF approach based on richness, we demonstrate improved patient stratification in bronchiectasis by identifying high frequency exacerbators with accuracy exceeding that of using a single microbial group. Hence to motivate and enable; researchers and clinicians to opt for an integrative strategy when analysing multi biome datasets, we developed a web tool “Integrative Microbiomics” (\url{https://integrative-microbiomics.ntu.edu.sg}) capable of implementing both SNF and weighted SNF to integrate microbiome datasets. This webtool also aims to motivate users to obtain multi biome datasets, as integrating datasets would better represent/ bring clarity to the underlying biological process. 

Limitations of this work include the cross-section design of the CAMEB cohort, a static dataset which we largely use to predict dynamic interaction \cite{Aogain2019,Mac1800766}. However, this is partially overcome by inclusion of a longitudinal case series to our analysis to better assess temporal dynamics in association to exacerbation and antibiotic treatment. Next, although 16S methodologies are well established, there are inherent limitations, including under-representation of mycobacteria, an important group of organisms in bronchiectasis \cite{Sulaiman2018}. Additionally, fungal ITS sequencing approaches are challenged by under-developed reference databases \cite{Ali2019}. Our virome analysis, while broad, comprehensive, and informed by established literature, targets a known virus panel and therefore is subject to bias. Nonetheless, this is partially addressed by our metagenomic dataset, which comprehensively assess the virome. While metagenomics potentially represents a less biased alternative approach, it underestimates fungal presence given the significantly higher airway bacterial burden hence obscuring the influence that fungi have on the interactome. We further acknowledge that sputum is an imperfect matrix, and, make no inference about lower airway ecology, noting only the clinical associations between sputum as a surrogate, readily obtainable, non-invasive upper airway sample. Finally, while observational data suggests potential causal association, other factors may drive observed effects. Observed interactions may represent epiphenomena of a selectively operating immune system, for example and our work did not include any assessment of host responses.